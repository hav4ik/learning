\documentclass[a4paper]{article}
\usepackage{../../remarks}

\title{"Introduction to Smooth Manifolds": Excercises and Notes}
\author{Chan Kha Vu}
\date{}

\begin{document}

\maketitle
\begin{abstract}
\textit{
    Hey, I'm Vu. This is just a place where I keep all of my notes while reading
    the book \href{https://www.springer.com/gp/book/9780387217529}{"John M. Lee: Introduction to Smooth Manifolds"}.
    I'm an old noob with very weak formal training in Mathematics, so lots of proofs here
    might look messy, too obvious and can be ommited, or even unstrict. Would be very
    grateful if you pointed them out to me tho.
    By the way, you can find more about me on my blog \href{https://hav4ik.github.io}{hav4ik.github.io}.
}
\end{abstract}


\section{Smooth Manifolds}
\bigskip

\begin{exercise}[1-1]
    We denote the equiv. class of $(x, 1)$ and $(x, -1)$ for $x \ne 0$ as $[x]$.
    Each point $x \in M$ has a n-hood $\{ [x+o] \}_{x-r}^{x+r}$ that is
    obviously euclidean by $(x, y) \equiv x$. The second countability follows trivially.
    It is not hausdorff because $0_a$ and $0_b$ cannot be separated by any
    neighbourhoods.
\end{exercise}

\begin{exercise}[1-2]
    Let $M = \bigcup_I M_{i\in I}$ where each $M_i$ is congruent to $\mathbb{R}$,
    then for each $x \in M_i$, it is locally euclidean and, since all $M_i$
    are disjoint, is also hausdorff. It is not second-countable as any base
    would be uncountable.
\end{exercise}

\begin{exercise}[1-3]
    $(\implies)$ By lemma 1.10, each manifold $M$ has a countable basis of precompact
    coordinate balls, so it is a countable union of compact subspaces
    ($\sigma$-compact).

    $( \impliedby )$ It suffices to show $X$ the space is second countable.
    Let $K_n$ be the covering of $X$ by compact sets, and assume they're
    all nonempty. For each $x\in K_n$, there is an open neighborhood
    $U_{x,n}$ which is homeomorphic to an open ball in $\mathbb{R}^m$.
    Then we have $K_n\subset \bigcup_{x\in K_n} U_{x,n}$. By compactness,
    there are finitely many $U_{x,n}$ which cover each $K_n$. By unioning these
    all together for each $n$ gives us a countable covering for $X$ by open sets
    which are homeomorphic to open balls in $\mathbb{R}^m$. Each of these sets
    has a countable basis. The countable union of these sets is also a countable
    set, so it remains to show that this set is a basis, which is easy.
\end{exercise}

\begin{exercise}[1-4]
\begin{enumerate}[label=(\alph*)]
    \item Since $\mathcal{U}$ is a cover of $M$, for each $x\in M$, there is an
        open n-hood $U_x \in \mathcal{U}$ that contains $x$. Since it intersects
        finitely many others, $\mathcal{U}$ is locally finite.
    \item Let's cover $\mathbb{R}^2$ with open sets
        $\left\{ V \right\} \cup \left\{ H_n \right\}_{n\in \mathbb{Z}}$, where
        $H_n = \left\{ (x, y) \mid n - \epsilon < y < n + \epsilon \right\}$,
        $\epsilon < \frac{1}{2}$, and
        $V = \left\{ (x, y) \mid - \epsilon < x < + \epsilon \right\}$.
        Such cover would be locally-finite, but $V$ actually intersects with
        infinitely many other subsets from collection $\mathcal{U}$.
    \item If each $U \in \mathcal{U}$ is precompact, then $\overline{U}$ is
        compact in $M$. For each $x \in \overline{U}$, let $V_x$ be the open
        n-hood of $x$ that intersects with finitely many elements from
        $\mathcal{U}$ (local finity). Since $\overline{U}$ is compact, from the
        cover $\left\{ V_x \right\}_{x \in \overline{U}}$ we can choose a finite
        subcover $V_{x^*}$, each element of which intersects with finite number
        of elements of $\mathcal{U}$. Then, since they all are finite cover of
        $U$, the element $U$ itself intersects with finitely many others from
        $\mathcal{U}$. Q.E.D.
\end{enumerate}
\end{exercise}

\begin{exercise}[1-5]
    $(\impliedby)$ By definition then, $M$ is a topological manifold, thus
    by proposition 1.15 it is paracompact and by proposition 1.11 it has
    countably many connected components.

    $(\implies)$ Let's examine a single connected component $M_0$ of $M$.
    Let $\left\{ (U_x, \phi_x) \right\}$ be the open cover of $M_0$ by open nbrh
    homeomorphic to $\mathbb{R}^n$. For each $U_x$, there is a countable
    basis of precompact coordinate balls, let's call it $\mathcal{B}_{U_x}$. The
    collection of such bases also forms a base $\mathcal{B}$ of topology of
    $M_0$. By paracompactness of $M_0$, $\mathcal{B}$ admits a locally-finite
    refinement, let's call it $\mathcal{B}'$, the elements of which are also
    precompact and forms a new cover of $M_0$.

    Since $M_0$ is locally-euclidean and connected, it is path-connected.
    For arbitrary pair $x, y \in M_0$ let's consider a path
    $\gamma \colon [0, 1] \to M_0$ where $\gamma(0) = x$ and $\gamma(1) = y$.
    Since $[0,1]$ is compact, codomain of $\gamma$ is also compact. So it can be
    covered with finite number of basis elements (precompact coord balls).
    Let's say $\delta(x,y) \in \mathbb{N}$ is the smallest size of all such covers,
    and $\delta(U, V) = \min_{x \in U, y \in V} \delta(x,y)$ for $U, V \in \mathcal{B}'$.
    In other words, from each element of $M_0$, we can reach another elements
    in a finite number of "hops".

    Let's fix an element $V_0 \in \mathcal{B}'$ and denote $\mathbb{V}_n$ as set
    of all $U \in \mathcal{B}' \colon \delta(U, V_0) = n$. It is easy to prove
    that $\mathcal{V}_n$ is finite $\forall n \in \mathbb{N}$. Then the collection
    $\mathcal{B}' = \left\{ \mathcal{V}_n \right\}_{n \in \mathbb{N}}$ is countable.
    Now, since $\mathcal{B}'$ is a refinement of $\mathcal{B}$, $M_0$ is
    second countable.

    Now, as the number of connected components of $M$ is countable, and each
    admits a countable base, the base of whole $M$ is also countable. (Q.E.D)
\end{exercise}

\begin{exercise}[1-6]
    First, let's follow the problem's hint and prove that $F_s \colon \mathbb{R}^n \to \mathbb{R}^n$,
    $F_s(x) = \left\| x \right\|^{s-1} x$ where $s>0$ is a homeomorphism but not a diffeomorphism
    unless $s=1$. $F_s$ sends each vector $x \in \mathbb{R}^n$ to a $x'$ that has
    the same direction as $x$ and $\left\| x' \right\| = \left\| x \right\|^s$, making
    it a bijection and, since it is clearly continuous, it is a homeomorphism.
    Either $F_s$ or the inverse $F_s^{-1} (x) = \left\| x \right\|^{\frac{1}{s} - 1} x$
    are not smooth at $x=0$ when $s \ne 1$. So $F_s$ is not a diffeomorphism.
    Moreover, $F_s \circ F_{s'}^{-1} = \left\| x \right\|^{\frac{s}{s'} - 1} x$ is not smooth, 
    unless $s' = s$.

    Let's get back to our problem. Let's assume $M$ has a smooth structure $\mathcal{A}_0$.
    From $\mathcal{A}_0$, we extract a paracompact cover of coordinate balls from its base,
    which forms a smooth atlas $\mathcal{A}'_0$. Let's consider all $(U, \varphi) \in \mathcal{A}'_0$
    such that $\exists p \in U \colon \varphi(p) = 0$ and $\forall (U', \varphi') \colon U\cap U' = \varnothing$.
    If there are none, we can manually construct one, as we will show later.
    By replacing such charts with $(U, F_s \circ \varphi)$, together with all
    other non-centered charts, we can form an smooth atlas $\mathcal{A}'_s$
    (it is an atlas because $F_s$ is a diffeomorphism on $\mathbb{R}^n \setminus \{0\}$),
    that is not smoothly compatible with $\mathcal{A}'_0$ because $F_s \circ \varphi \circ \psi^{-1} $ is not smooth, unless $s=1$.
    By proposition 1.17, $\mathcal{A}'_0$ belongs to a different smooth structure,
    let's call it $\mathcal{A}_s$. Similarly, we can prove that for any $s>0$ and $s'>0$
    that are not equal to $1$, the transitions $F_s \circ \varphi \circ \varphi^{-1} \circ F_{s'}^{-1} = F_s \circ F_{s'}^{-1} $
    is not smooth, so $\forall s,s'>0 \colon s \ne s'$, $\mathcal{A}_s$ and
    $\mathcal{A}_{s'}$ are not smoothly compatible, thus belongs to different
    smooth structures. Clearly, their number is uncountable.

    Let's show that we can construct an atlas that contains charts centered at
    zero and the center of such charts does not contain in any other charts.
    Get an arbitrary $(U, \varphi)$,
    get an paracompact element $V$ from the local basis such that $\overline{V} \subset U$.
    Since it is closed, $M \setminus \overline{V}$ is open, so is its intersection
    with all other basis elements. The induced topology on $M \setminus \overline{V}$,
    together with the chosen $(U, \varphi)$, forms an atlas smoothly compatible
    with $\mathcal{A}$, which we will also call $\mathcal{A}'_0$. (Q.E.D).
\end{exercise}

\begin{exercise}[1-7]
(a) and (b) are trivial trigonometry, so I wouldn't prove them here.
\begin{enumerate}[topsep=0pt]
    \item[(c)] On charts $\mathbb{S}^n \setminus \{N\}$ and $\mathbb{S}^n \setminus \{S\}$,
        the transition map $\widetilde{\sigma} \circ \sigma^{-1} = -\sigma \circ -\boldsymbol{1} \circ \sigma$
        sends each point $(x_1, \ldots, x_n, x_{n+1})$ to $(x_1, \ldots, x_n, -x_{n+1})$ (mirror symmetry),
        which is a diffeomorphism. It covers $\mathbb{S}^n$, so it is an atlas.
        This atlas defines a unique smooth structure, according to proposition 1.17.
    \item[(d)] It is sufficient to show that $(\mathbb{S}^n \setminus \{N\}, \sigma)$ and
        the chart $\varphi_{i}(x) = (x_1, \dots, x_{i-1}, x_{i+1}, \dots, x_n)$
        are smoothly compatible. In fact, $\sigma \circ \varphi_i^{-1} $ and
        $\varphi_i \circ \sigma^{-1} $ are just reprojections, which are smooth.
\end{enumerate}
\end{exercise}

\begin{exercise}[1-11]
    $M = \overline{\mathbb{B}}^n = \mathbb{B}^n \cup \mathbb{S}^{n-1}$. Hausdorffness
    and second countability follows naturally as $M \in \mathbb{R}^n$. Now every
    point $x \in \mathbb{B}^n$ admits an open nbh $U_x = \mathcal{B}_{1-\|x\|}(x)$
    --- a precompact n-ball that contained in $\mathbb{B}^n$. We pair each such
    nbhs with a natural homeomorphism $\varphi_x \doteq \boldsymbol{1}$ (which
    will be our inner chart).
    For each $x \in \partial\overline{\mathbb{B}}^n = \mathbb{S}^{n-1}$, we construct a chart
    $(U_x, \varphi_x)$ as following. $U_x = \mathcal{B}_{\epsilon}(x) \cap \overline{\mathbb{B}}^n$,
    with $\epsilon$ chosen such that it is much smaller than the hemishpere (precompact coordinate ball in the basis of induced topology).
    We then choose $\varphi_x = \pi_{\overline{\mathbb{HS}}^{n} \mid \mathbb{H}^n} \circ \pi_{\overline{\mathbb{B}}^n \mid \overline{\mathbb{HS}}^n}$.
    Geometrically, $\pi_{\overline{\mathbb{B}}^n \mid \overline{\mathbb{HS}}^n}$ projects
    our nbh $U_x \subseteq \overline{\mathbb{B}}^n$ to a closed unit hemisphere in higher dimension
    (upwards vertically). Then, $\pi_{\overline{\mathbb{HS}}^{n} \mid \mathbb{H}^n}$
    projects that surface horizontally back to $\mathbb{H}^n$. Proving that this
    combination is a diffeomorphism $U_x \to \mathbb{H}^n$ is trivial. From this,
    it follows that our chart is a boundary one. Transition maps between internal
    charts is $\boldsymbol{1}$, which makes our interior atlas smoothly compatible
    with standard structure. Transition map between boundary charts $V$ and $V'$
    are just a combination of diffeomorphic projections, which itself is diffeomorphism.
    Transition maps between interior chart and boundary are also identity maps
    and diffeomorphic projections. So, Our smooth atlas is compatible with standard one.
\end{exercise}

\begin{exercise}[1-12]
    By example 1.8, the product of manifolds is also a manifold of dimension
    equal to the sum of dimensions of these manifolds. So, let $M = M_1 \times \dots \times M_k$
    be a $n = \sum_{i=1}^k n_i$ dimensional manifold, where $n_i$ is the dimensionality
    of $M_i$. Then, we can treat the problem as showing $M \times N$ is a manifold with
    boundary. $M \times N = M\times \text{Int}N \cup M\times \partial N$. Now,
    $M \times \text{Int}N$ is a $n\times m$ manifold ($m$ is the dimensionality of $N$).
    The boundary $M \times \partial N$ is a manifold with boundary with dimension
    $n \times (m-1)$.

    Let's show that each point $x \in M\times \partial N$ is a boundary point.
    Consider a chart $U_x = U^M_x \times U^N_x$. By definition, $U^N_x$ is a
    boundary chart of N, so it is homeomorphic to $\mathbb{H}^m$. Then $U_x$ is
    homeomorphic to $\mathbb{R}^n \times \mathbb{H}^m = \mathbb{H}^{n+m}$. Thus,
    by proposition 1.37, $M \times \partial N$ is the boundary of $M \times N$. (Q.E.D)
\end{exercise}




\section{Smooth Functions and Smooth Maps}

\begin{note}[on ex. 2.1]
    This one is obvious but I'm noob, so I'll state it clear. We need to prove that $C^\infty(M)$ forms a commutative ring, and is a commutative
    and associative algebra over $\mathbb{R}$. The addition $f+g$ satisfies
    axioms of abelian group (namely \textit{closure, associativity, commutativity,
    additive identity}, and \textit{additive inverse}), and the pointwise multiplication
    $f \cdot g$ satisfies monoidal axioms (namely \textit{associativity} and
    \textit{multiplication identity}). The pointwise  \textit{left-} and
    \textit{right-distributivity} of multiplication with respect to addition are also
    satisfied. Thus, $C^\infty(M)$ is a commutative ring. The pointwise multiplication
    is also \textit{bilinear}, together with other properties of the product
    makes $C^\infty(M)$ a associative and commutative algebra.
\end{note}

\begin{note}[on ex. 2.7]
    First, let's prove the proposition 2.5.
    \begin{itemize}[topsep=0pt]
        \item[(a)] $\implies$ By proposition 2.4, $F$ is continuous, so $F^{-1} (V)$ is open in $M$,
            thus in $U \cap F^{-1} \left( V \right) $. The composite map is
            smooth by definition.

        $\impliedby$ $F\mid_{U \cap F^{-1} (V)} = \varphi \circ \left( \psi \circ F \circ \varphi^{-1} \right) \circ \psi^{-1}$
            is smooth by definition. So, at least on a refinement of charts
            $(U, \varphi)$ and $(V, \psi)$, which are also smooth charts, $F$ is smooth.

        \item[(b)] $\implies$ Applying (a) to every points and every pair of charts
            $U_\alpha$ and $V_\beta$ that nontrivially intersects the image $F(U_\alpha)$,
            we get a formulation of (b). The other way ($\impliedby$) is obvious,
            since it implies (a).
    \end{itemize}
    Proposition 2.6 is quite trivial, and can be implied just by definitions.
\end{note}


\end{document}
