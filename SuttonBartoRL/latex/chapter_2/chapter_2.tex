\documentclass[a4paper]{article}
\usepackage[left=20mm, right=20mm, top=20mm, bottom=20mm]{geometry}
\usepackage{amsthm, amsmath, amssymb}
\usepackage[framemethod=TikZ]{mdframed}

\newenvironment{exercise}[1][]{%
  \mdfsetup{frametitle={%
    \tikz[baseline=(current bounding box.east),outer sep=0pt]
    \node[anchor=east,rectangle,fill=blue!20]
    {\strut Exercise~#1};}}%
  \mdfsetup{
    innertopmargin=5pt,linecolor=blue!20,%
    linewidth=2pt,topline=true,%
    frametitleaboveskip=\dimexpr-\ht\strutbox\relax}
  \begin{mdframed}[]\relax}
  {\end{mdframed}}

\newenvironment{note}[1][]{%
  \mdfsetup{frametitle={%
    \tikz[baseline=(current bounding box.east),outer sep=0pt]
    \node[anchor=east,rectangle,fill=red!40]
    {\strut Note~(#1)};}}
  \mdfsetup{
    innertopmargin=5pt,linecolor=red!40,%
    linewidth=2pt,topline=true,%
    frametitleaboveskip=\dimexpr-\ht\strutbox\relax}
  \begin{mdframed}[]\relax}
  {\end{mdframed}}

\title{Notes on Chapter 2}
\author{Chan Kha Vu}
\date{}

\begin{document}
\maketitle

\begin{note}[page 30]
  \noindent

  The claim that \(\sum_a \frac{\partial \pi(a)}{\partial H_t(b)} = 0\), where $\pi_t(a)$ is softmax function over arguments $H_t(b)$, is not obvious at all.
  So, let's prove it.

\begin{align*}
  \frac{\partial \pi_t(a)}{\partial H_t(b)}
  &\doteq \frac{\partial }{\partial H_t(b)} \frac{e^{H_t(a)}}{\sum_i e^{H_t(i)}}
  \tag{$\frac{\partial fg}{\partial x} = \frac{\partial f}{\partial x} g + \frac{\partial g}{\partial x} f$}
  \\ &=
  \left( \frac{\partial}{\partial H_t(b)} e^{H_t(a)} \right) \cdot
  \left( \frac{1}{\sum_i e^{H_t(i)}} \right)
  +
  \left( \frac{\partial}{\partial H_t(b)} \frac{1}{\sum_i e^{H_t(i)}} \right) \cdot
  e^{H_t(a)}
  \tag{$\frac{\partial }{\partial x} \frac{1}{f} = -\frac{1}{f^2} \frac{\partial f}{\partial x} $}
  \\ &=
  \frac{\boldsymbol{1}_{a=b} \cdot e^{H_t(a)}}{\sum_i e^{H_t(i)}}
  -
  \frac{e^{H_t(b)}}{\left(\sum_i e^{H_t(i)}\right)^2} \cdot
  e^{H_t(a)}
  \\ &=
  \frac{\boldsymbol{1}_{a=b} \cdot \sum_i e^{H_t(i)} e^{H_t(a)} - e^{H_t(b)} e^{H_t(a)}}{\left(\sum_i e^{H_t(i)}\right)^2 }
  \tag{$\sum_i e^{H_t(i)}$ is constant}
\end{align*}

Now, let's use the fact that $\pi_t(a)$ sums to one, because it is a softmax function. Now, the claim becomes pretty transparent:

\begin{align*}
  \sum_a \frac{\partial \pi_t(a)}{\partial H_t(b)}
  &=
  \sum_a \frac{\boldsymbol{1}_{a=b} \cdot \sum_i e^{H_t(i)} e^{H_t(a)} - e^{H_t(b)} e^{H_t(a)}}{\left(\sum_i e^{H_t(i)}\right)^2 }
  \\ &=
  \frac{\sum_i e^{H_t(i)} e^H_t(b)}{\left( \sum_i e^{H_t(i)} \right)^2} -
  \frac{\sum_a e^{H_t(b)} e^{H_t(a)}}{\left( \sum_i e^{H_t(i)} \right)^2}
  = 0
  \tag{Q.E.D.}
\end{align*}
\end{note}


\end{document}
